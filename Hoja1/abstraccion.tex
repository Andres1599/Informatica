\documentclass{article}
\begin{document}

    \title{Hoja de trabajo No. 1, Informática 2}
    \author{Anibal Andres Higueros Hernández}
    \maketitle
    
    \section{Que Hacer}
    \begin{enumerate}
        \item Lista de que hacer:
        \begin{itemize}
            \item Nombre de la actividad
            \item Tiempo empleado en la actividad
            \item Numero de personas requeridas para la actividad
            \item Utencilios (una variable boolean para saber si se estan utilizando)
            \item Estado de la actividad
            \item Fecha de la actividad
        \end{itemize}
    \end{enumerate}
    \section{Que Haceres}
    \begin{enumerate}
       \item Lista de Haceres
       \begin{itemize}
           \item agregar: función la cual carga un nuevo elemento a la lista.
           \item eliminar: retira un elemento dentro de la lista.
           \item contador: cuanta la cantidad de elementos dentro de la lista.
           \item buscar por index: busca dentro de una lista de acuerdo a un index específico.
           \item item: para obtener o establecer un elemento en un indice especificado
       \end{itemize}
    \end{enumerate}
\end{document}