\documentclass{article}
    \usepackage[utf8]{inputenc}
    \documentclass[10pt,a4paper]{article}
    \usepackage[utf8]{inputenc}
    \usepackage{amsmath}
    \usepackage{amsfonts}
    \usepackage{amssymb}
    \begin{document}
    
    \title{Genericos}
    \author{Aníbal Higueros, Universidad del Istmo}
    \date{ 22 / Febrero / 2018 }
    
    \begin{document}
    
    \maketitle
    
    \section{Ventajas de los genericos}
    
    Las ventajas que tienen los métodos de carácter generico sobre métodos que reciben como parametro a un objeto son las siguientes:
    \begin{itemize}
        \item Traspazan la carga de seguridad al compilador.
        \item No hay necesidad de escribir codigo para comprobar que el tipo de dato es el correcto, puesto que este se establece en el tiempo de compilacion.
        \item No hay necesidad de heredar de un tipo de base y cambiar o remplazar miembros.
        \item Los tipos de colección genéricos suelen conseguir mejor rendimiento para guardar y controlar tipos de valor porque no hay necesidad de realizar conversiones de los tipos de valor.
        \item Los delegados genéricos habilitan las devoluciones de llamada con seguridad de tipos sin la necesidad de crear varias clases de delegado.
    \end{itemize}
    \end{document}
    \end{document}
    