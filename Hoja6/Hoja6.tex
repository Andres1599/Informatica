\documentclass{article}
\usepackage[utf8]{inputenc}
\usepackage{mathrsfs}
\usepackage[spanish]{babel}
\usepackage{mathdots} 
\usepackage{mathdots}

\title{Hoja6}
\author{Andr�s Higueros Hern�ndez}
\date{25 Marzo 2018}

\begin{document}

\maketitle

\section{Primera Parte}
    Para poder hacerlo necesito verificar que alguno de los numeros no sea cero por lo tanto utilizare una funcion llamada esCero(Numero $a$)
    \begin{enumerate}
        \item  funcion Suma(Numero $a$,Numero $b$){ $:=$\hfill \break
            si esCero($a$) retorno $b$,\hfill \break
            si esCero($b$) retorno $a$,\hfill \break
            Numero $c$ es el predeccesor del numero $a$
            de lo contrario retorno un nuevo numero que es igual  a Suma($c$,$b$)\hfill \break
        }
        \hfill \break
        - en terminos de la funcion MayorQue() se declara la funcion Multiplicar
        \item funcion Multiplica(Numero $a$, Numero $b$) $:=$\hfill \break
        Numero nAcumulado $:=$ nuevo numero,\hfill \break
        si esCero($a$) o esCero($b$) retornar cero,\hfill \break
        de lo contrario retornar\hfill \break
        si MayorQue($b > 1$) $:=$ Suma($-a$,Multiplica($a$,$b.Predecesor$))\hfill \break
        - se necesita de una funcion de saver si los numeros son iguales SonIguales(Numero $a$,Numero $b$)\hfill \break
        //una funcion que retorna un valor booleano
        \item funcion MayorQue(Numero $a$,Numero $b$)$:=$ \hfill \break
        si SonIguales($a$,$b$) retornar falso, \hfill \break
        de lo contrario: \hfill \break
        si SonIguales($a$,SonIguales($a$,$b +1 $) retorno falso \hfill \break
        de lo contrario retorno verdadero
    \end{enumerate}
\section{Segunda Parte}
    \begin{enumerate}
        \item 
    \end{enumerate}
\end{document}
